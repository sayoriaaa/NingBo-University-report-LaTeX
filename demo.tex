\documentclass{nbueecs}
% =============================================
% Part 0 Edit the info
% =============================================

\major{20计算机科学与技术}
\name{sayoriaaa}
\title{信息科学与工程学院\\课程设计报告}
\stuid{206001870}
\date{\zhtoday}
\course{硬件技术基础设计课程}
\instructor{钱江波}
\expname{单周期 CPU 逻辑设计}
\exptype{设计实验}
\partner{Bob}
\starttime{2022年9月4日}
\finishtime{2022年11月12日}


\begin{document}
% =============================================
% Part 1 Header
% =============================================
\makecover

\begin{abstract} 
对本次课程设计的任务和要求进行简要描述。重点介绍自己设计的指令类型、模块和测试结果。至少200字以上。
\par\noindent\textbf{关键词:}宁波大学实验报告;\LaTeX;单周期CPU;Quartus
\end{abstract}

\clearpage
\tableofcontents{}

% =============================================
% Part 2 Main document
% =============================================
\clearpage
\section{设计的目的与目标}
\subsection{设计目的}
xxxxx
\subsection{设计目标}
xxxxx

\section{课程设计器材}
\subsection{硬件平台}
xxxxx
\subsection{软件平台}
xxxxx

\section{CPU逻辑设计总体方案}
\subsection{指令模块}
xxxxx
\subsection{器件单元模块}
xxxxx

\section{模块详细设计}
\subsection{指令设计模块}
xxxxx
\subsection{器件单元设计模块}
xxxxx

\section{测试结果及分析}
\subsection{测试方案}
xxxxx
\subsection{模块初始值}
xxxxx
\subsection{测试结果及分析}
xxxxx

\section{心得与体会}

\clearpage
\setcounter{secnumdepth}{0} % 抑制section的编号
\begin{thebibliography}{99}\label{sec:bib}
\addtolength{\itemsep}{-1.5ex} % 缩小行间距,可选
\bibitem{QJB}钱江波. 短学期指导. 宁波大学实验手册,2016/09/10.
\footnote{钱江波,男,1974年生,教授,计算机应用技术博士。目前主持国家自然科学基金、浙江省自然科学基金、宁波市自然科学基金各1项,主要涉及数据库、数据流及硬件技术研究与开发。在国内外高水平的学术期刊发表论文多篇,如“Journal of Circuits, Systems, and Computers”、“电子学报”、“计算机研究与发展”等。同时也从事数据库和硬件相关技术的教学工作,包括“计算机组成与结构”、“汇编语言与接口技术”等。电子邮箱:qianjb@163.com}.
\end{thebibliography}


\end{document}
